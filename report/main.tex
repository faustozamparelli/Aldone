\documentclass{article}

\usepackage[english]{babel}
\usepackage{fouriernc}

\usepackage{hyperref}

\nocite{*}

\title{RemAInder}
\author{Fausto Zamparelli, Daniel Falbo}
\date{May 26, 2024}

\hypersetup{
colorlinks=true,
linkcolor=cyan,
filecolor=magenta,      
urlcolor=blue,
pdftitle={Overleaf Example},
pdfpagemode=FullScreen,
}

\begin{document}

\maketitle


\section*{Introduction}
To build this project, we solved various tasks independently and then made them communicate between each other for the final product.

[1] \href{https://humane.com/media/cosmos-an-operating-system-for-the-ai-era}{Cosmos: An Operating System for the AI Era}

[2] aiXplain


\section*{Method}


\subsection*{Text-to-Speech task}

\subsection*{Speech-to-Text task}

\subsection*{Todo-extraction task}

\subsection*{Todo category classification task}

[Confusion Matrix here]

\subsection*{Vision Food Classifier task}

We first asked the LLM model to give us the entire updated grocery list object, but empirically it often made errors. What we ended up doing is asking the LLM model for indices of the grocery list that were updated, and then we updated the grocery list object ourselves. This way, we could ensure that the grocery list object was updated correctly.


\section*{Results}

\subsection*{Glue Orchestration}


\end{document}